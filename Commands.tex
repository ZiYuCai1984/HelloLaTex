\documentclass[a4paper,11pt,titlepage]{ltjsarticle} %titlepage:タイトルと本文(目次含む)のページ分割コマンド

\usepackage{type1cm} %フォントサイズの制限解除
\usepackage{indentfirst} %字下げの不具合解消
\usepackage{ulem} %\underline{}
\usepackage{hyperref} %\url
\makeindex %部節段落のデータベース化
\parindent = 0pt %改行時の字数下げ幅の指定 指定無しだと通常(一文字)の字下げ


\begin{document}


\title{title} %タイトルを入力
\author{author} %著者名を入力
\date{date} %日付を入力 \todayはコンパイル時の年月日
\maketitle %タイトル作成

\tableofcontents %目次自動作成


\newpage %強制改ページコマンド

%ここから本文
\part{partname} %部(省略可能)
\section{sectionname} %節
\subsection{subsectionname} %小節
\subsubsection{subsubsectionname} %小小節
\paragraph{paragraphname} %段落
\subparagraph{subparagraphname} %小段落

\underline{linedtext}
\footnote{notetext} %\footnote[number]{notetext} [number]省略可
nonlinedtext

From \cite{keyname1}: %\cite[text]{keylist} [text]省略可
\begin{quote}
  hogehoge said "fugafuga."\\
  \emph{fugafuga said "HOGEHOGE!"}\\ %強調*jsarticleなど,日本語環境では下線;英語環境では斜線
\end{quote}


\begin{thebibliography}{99} %参考文献データベース&表示 {}中の数字は文献のナンバリングの上限数
  \bibitem{keyname1} author『booktitle』(Publisher、YYYY年)
  \bibitem{keyname2} 『pagetitle』(\url{URL})「sitename」(YYYY年M月D日閲覧)
\end{thebibliography}


\end{document}
